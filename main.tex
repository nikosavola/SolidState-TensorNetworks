\input{preamble.tex}
\addbibresource{references.bib}
\hypersetup{
    pdftitle={Machine learning with many-body tensor networks | Niko Savola},
}
%   header
\lhead{\textsf{Special exercise}}
\chead{\textsf{PHYS-E0421 - Solid-State Physics}}
\rhead{\textsf{Niko Savola \textbf{653732}}}
%   footer
\lfoot{}
\cfoot{}
\rfoot{\thepage}


\usepackage[braket, qm]{qcircuit}  % Qiskit output


\begin{document}

\begin{titlepage}
    {\sffamily
    \noindent
    \fontsize{12}{14}\selectfont
    Aalto University \newline
    School of Science \newline
    Department of Applied Physics

    \vspace{40mm}

    \noindent
    \fontsize{14}{16}\selectfont
    \emph{Niko Savola}

    \vspace{10mm}

    \noindent
    \fontsize{18}{22}\selectfont
    \textbf{Machine learning with many-body tensor networks}\\

    \fontsize{12}{14}\selectfont
    \noindent
    Submitted for approval: \today

    \vspace{70mm}

    \noindent
    Special exercise \\[4mm]
    PHYS-E0421 \textendash{} Solid-State Physics \\[4mm]
    } % end of \sffamily

\end{titlepage}
\newpage


\section{Introduction}

TODO review paragraph.
https://youtu.be/q8UTwdjS95k

Central to these developments are
tensor-network representations of many-body quantum states.
These are successful variational families of many-body wave
functions, naturally emerging from low-entanglement representations of quantum states (Verstraete, Murg, and Cirac,
2008). T

\section{Theory}

\subsection{Tensor networks}

TODO notation from https://arxiv.org/pdf/1905.01330.pdf

\subsection{Modelling many-body physics}

todo work as Ansätze


\subsection{Numerical implementation}


https://quimb.readthedocs.io/en/latest/examples/ex_tn_train_circuit.html


\cite{Roberts2019}


\section{Results}

todo


\begin{figure}[htb]
    \centering

    \includegraphics[width=0.97\textwidth]{figures/ansatz_circuit.pdf}

    \caption{<caption>}
    \label{fig:qasm_circuit}
\end{figure}


\section{Summary}

Todo





TODO https://github.com/google/TensorNetwork

https://arxiv.org/pdf/1905.01331.pdf

%-------------------
%   Bibliography
%-------------------

\newpage
\pagestyle{plain}
%\setlength\bibitemsep{1.3\itemsep}
%\setlength\bibitemsep{0.1\baselineskip}
\addcontentsline{toc}{section}{Viitteet}
\renewcommand*{\bibfont}{\footnotesize}
\printbibliography{}


\end{document}

\documentclass[a4paper,12pt]{article}           %   A4, pt changes regular font size
\frenchspacing                                  %   Finnish spacing
\usepackage{a4wide}                             %   Less margis
\usepackage[british]{babel}                     %   Edit this to match your language, e.g., finnish
\usepackage{xcolor}                             %   color support
    \definecolor{my_grey}{HTML}{7F7F7F}
\usepackage{icomma}                             %   Better comma in eqs. if used
\usepackage{hyperref}                           %   Clickable URLs
\usepackage{booktabs}                           %   Better tables
\usepackage{siunitx}                            %   Physics unit stuff
\usepackage{listings}                           %   Better tables
\usepackage{graphicx}                           %   Figures work
\usepackage{fontspec}                           %   Better way to handle custom fonts
\usepackage{amsmath}                            %   Basic math
\usepackage{esint}                              %   Various fancy integral symbols
\usepackage[capitalise]{cleveref}               %   \cref better than \ref
    \crefname{section}{Sec.}{Secs.}             %   Add abbreviation for Sections
\usepackage[position=top]{subfig}
    \captionsetup[subfigure]{position=top, labelfont=bf,textfont=normalfont,singlelinecheck=off,justification=raggedright}
\captionsetup[subfigure]{subrefformat=simple,labelformat=simple,listofformat=subsimple}
\renewcommand\thesubfigure{(\alph{subfigure})}
\usepackage{sectsty}                            %   Edit section titles
    %\allsectionsfont{\normalfont\sffamily}     %   ALL titles use sans-serif
    \sectionfont{\normalfont\sffamily\LARGE\color{my_grey}} 
    \subsectionfont{\normalfont\sffamily\large\color{my_grey}} 
    \subsubsectionfont{\normalfont\sffamily\normalsize\color{my_grey}}
\setsansfont{URW Gothic L}[
    Path = fonts/,
    Extension =.ttf,
    UprightFont = * Regular,
    ItalicFont = * Italic,
    BoldFont = * Bold,
    BoldItalicFont= * Bold Italic]
\setmonofont{Source Code Pro}[
    Path = fonts/,
    Extension =.ttf,
    UprightFont = *-Regular,
    ItalicFont = *-Italic,
    BoldFont = *-Bold,
    BoldItalicFont= *-BoldItalic
]
\setmainfont{AdobeTextPro}[
    Path = fonts/,
    Extension =.ttf,
    UprightFont = *-Regular,
    ItalicFont = *-It,
    BoldFont = *-Bold,
    BoldItalicFont= *-BoldIt]
\usepackage[libertine]{newtxmath}               %   Matching math font
\usepackage{fancyhdr}                           %   Nice header and footer header (and/or footer)
    \pagestyle{fancy}                               %   Makes package work
\usepackage[section]{placeins}                  %   Figures stay in declared section
\usepackage{minted}                             %   Beautiful verbatim code
    \usemintedstyle{emacs}                          %   code style
    \setminted{
        frame=lines,
        framesep=2mm,
        baselinestretch=0.9,
        fontsize=\footnotesize,
        linenos,
        breaklines
    }
\usepackage{csquotes}                           %   Correct quotes for babel
\usepackage[backend=biber,
            style=ieee,
            urldate=long,
            maxnames=5,
            dateabbrev=false]{biblatex}         %   Citation style with biblatex
\DeclareSourcemap{                              %   No ISSN for journals
    \maps[datatype=bibtex]{
        \map{
        \step[fieldset=issn, null]
        }
    }
}
\renewbibmacro*{doi+eprint+url}{                %   Print URL iff no doi
    \printfield{doi}
    \newunit\newblock{}
    \iftoggle{bbx:eprint}{
        \usebibmacro{eprint}
    }{}
    \newunit\newblock{}
    \iffieldundef{doi}{
        \usebibmacro{url+urldate}}
        {}
}

\hypersetup{
    pdfauthor={Niko Savola},                    %   Author here
}

\linespread{1.1}
\setlength{\headheight}{15pt}                   %   Suppress height warnings
\addtolength{\topmargin}{-2.4pt}


% Set commonly used math commands here
\newcommand*{\pd}[3][]{\ensuremath{\frac{\partial^{#1} #2}{\partial #3}}}
\newcommand*{\dt}[3][]{\ensuremath{\frac{\textrm{d}^{#1} #2}{\textrm{d} #3}}}
\newcommand{\dbar}{\textrm{d}\hspace*{-0.08em}\bar{}\hspace*{0.1em}}
\newcommand{\de}{\textrm{d}}
\newcommand{\sref}[1]{\textbf{\small\subref{#1}}}
\newcommand{\ve}[1]{\textbf{#1}}
\newcommand{\e}{\mathrm{e}}
